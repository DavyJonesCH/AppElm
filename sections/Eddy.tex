\section{Eddy Currents}
When an $A_\phi$-Field is going perpendicular through an object, it induces eddy currents. In a transformer this means that the inner LV foil has the highest eddy currents, therefore, it would make sense to make this foil thicker. \newline

Due to skin-effect the current distribution in the LV-foils is non-uniform and can be computed as
\begin{equation*}
	P_\textrm{LV AC} > P_\textrm{LV DC} = R_\textrm{LV DC} \cdot I^2
\end{equation*}

\textbf{\\Shielding \\ \\} 
If there are components made of steel (e.g. a clamp) they can be shielded with copper in order to reduce eddy current. \newline 
\begin{tabular}{ll}
	Steel: 	& \(\displaystyle \mu_r = 150\) \\
			& \(\displaystyle \sigma_{Fe} = 5 \cdot 10^6 \frac{S}{m} \) \\
	Copper: & \(\displaystyle \mu_r 0= 1 \) \\
			& \(\displaystyle \sigma_{Cu} = 5.9 \cdot 10^7 \frac{S}{m} \) \\
\end{tabular} \\ \\
\begin{tabular}{lll}
	 & & Fe:Cu \\
	1. Step: Induced E-field & \(\displaystyle \nabla \times \vec{E} = j\omega\color{red}\mu\color{black}\vec{H} \) & 150:1 \\
	2. Step: Eddy currents  & \(\displaystyle J = \color{red}\sigma\color{black}\vec{E} \) & 1:10 \\
	\textbf{Total:} & & \textbf{15:1}
\end{tabular}\\ \\
$\Rightarrow$ \textbf{15x less eddy current and less losses!}