\section{FDTD}
The Finite Difference Time Domain (FDTD) is a method to numerically solve Maxwell equations in time domain, therefore, it is especially suited for transient analysis and/or for wideband analysis. Generally, it is easy to implement and it scales very good for large simulations, however, it is difficult to simulate curved surfaces with this approach.

\subsection{Numerical Solution of PDEs}
The general Taylor series expansion of function $f(x,t)$ can be written in numerical form as 
\begin{equation*}
	{f_i}^{n+1}= {f_i}^n + \underbrace{\left(t^{n+1} - t^n\right)}_{\Delta t} \frac{\partial f}{\partial t} \bigg\rvert_{i}^{n} + \frac{1}{2} + \underbrace{\left(t^{n+1} - t^n\right)}_{\left(\Delta t\right)^2} \frac{\partial^2f}{\partial t^2} \bigg\rvert_{i}^{n} \Rightarrow \frac{\partial f}{\partial t} \bigg\rvert_{i}^{n} = \frac{{f_i}^{n+1}-{f_i}^n}{\Delta t} - \underbrace{\frac{1}{2}\Delta t \frac{\partial ^2 f}{\partial t^2}\bigg\rvert_{i}^{n}}_{O(\Delta t)} - \dots
\end{equation*}
This approximation is called first order accurate forward difference approximation. \newline Using a sample in the past and a sample in the future it is possible to increase the precision of the approximation (residual error scales down with $\Delta t^2$)
\begin{equation*}
	\frac{\partial f}{\partial t} \bigg\rvert_{i}^{n} = \frac{{f_i}^{n+1}-{f_i}^{n-1}}{\Delta t} - \underbrace{\frac{1}{6}\left(\Delta t\right)^2 \frac{\partial ^3 f}{\partial t^3}\bigg\rvert_{i}^{n}}_{O\left[(\Delta t)^2\right]} - \dots
\end{equation*}
Following the same concept, one can calculate the derivative at half-time-steps:
\begin{equation*}
	\frac{\partial f}{\partial t} \bigg\rvert_{i}^{n+\frac{1}{2}} = \frac{{f_i}^{n+1}-{f_i}^{n}}{\Delta t} - \underbrace{\frac{1}{24}\left(\Delta t\right)^2 \frac{\partial ^3 f}{\partial t^3}\bigg\rvert_{i}^{n}}_{O\left[(\Delta t)^2\right]} - \dots
\end{equation*}
With the same approach the second order centered approximation for the spatial derivation can be written as
\begin{equation*}
	\frac{\partial f}{\partial t} \bigg\rvert_{i}^{n} = \frac{{f_{i+1}}^{n}-{f_{i-1}}^{n}}{2\Delta x} - \underbrace{\frac{1}{6}\left(\Delta t\right)^2 \frac{\partial ^3 f}{\partial x^3}\bigg\rvert_{i}^{n}}_{O\left[(\Delta x)^2\right]} - \dots
\end{equation*}

\subsection{Interlieved Leapfrog (lossless transmission line in this case)}
\begin{minipage}[rt]{9cm}
	\begin{tabular}{l}
		By discretising the voltage at integer grid points\\ (centered at $n+1/2,i$), whilst the current at half grid \\points (centered at $n,i+1/2$): \\
		\begin{equation*}
			V_{i}^{n+1} = V_{i}^{n} - \left(\frac{\Delta t}{C \Delta x}\right) \left[I_{i+1/2}^{n+1/2} - I_{i-1/2}^{n+1/2}\right]
		\end{equation*} \\
		\begin{equation*}
			I_{1+1/2}^{n+1/2} = I_{1+1/2}^{n-1/2} - \left(\frac{\Delta t}{L \Delta x}\right) \left[V_{i+1}^{n} - V_{i}^{n}\right]
		\end{equation*} \\
		Stability for 1D Leapfrog (CFL): \\
		\begin{equation*}
			\Delta t \leq \frac{\Delta x}{|v_p|} \textrm{ where } v_p = \frac{1}{\sqrt{LC}}
		\end{equation*} \\
		General stability (CFL) for dimension $D$ \\if $\Delta x = \Delta y = \Delta z$: \\
		\begin{equation*}
			\Delta t \leq \frac{\Delta x}{\sqrt{\varepsilon \mu}} \cdot \frac{1}{\sqrt{D}} = \frac{\Delta x}{v_p} \cdot \frac{1}{\sqrt{D}}
		\end{equation*}
	\end{tabular}
\end{minipage}
\begin{minipage}[rt]{10cm}
	\includegraphics[width=.95\textwidth]{./images/leapfrog.pdf}
\end{minipage}

\subsection{3D Derivation}
\begin{minipage}{12cm}
	The Yee Cell is organized in such a way: \\
	\begin{itemize}
		\item The origin of each cell defines the generic point (i.e. $i,j,k$) at which all vectors are referred to.
		\item The specific component of the electric field is always defined on the \textbf{edges} of the cell containing the origin, displaced by ½ position.
		\item The specific component of the magnetic field is always defined on the \textbf{face} of the cell containing the origin, exactly in the middle.
	\end{itemize}
	Advantages of using the Yee Grid: \\
	\begin{itemize}
		\item Curl equations are automatically satisfied
		\item When considering source free media and second order centered differencing, it can be shown that the Yee cell guaratnies divergence difference equations \(\displaystyle \left(\nabla \cdot \vec{B} = 0 \textrm{ (always true)}, \nabla \cdot \vec{E} = 0 \textrm{ (normally true)} \right)\) 
		\item Boundary donditions are automatically satisfied
	\end{itemize}
\end{minipage}
\begin{minipage}{7cm}
	\begin{flushright}
	\includegraphics[width=.95\textwidth]{./images/Yee.pdf}
	\end{flushright}
\end{minipage}

\subsection{Normalization of H}
In free space Electric and Magnetic field are related through the characteristic impedance $Z_0$ of free space with
\begin{equation*}
	Z_0 = \sqrt{\frac{\mu_0}{\varepsilon_0}} \approx 120\pi~\Omega \approx 377~\Omega.
\end{equation*}
This means that the amplitude of the electric field is always at least 377 times bigger than the one of the magnetic field. In order to have a better numeric approximation is common practice to normalize the magnetic field vector using the following expression
\begin{equation*}
	^*\overline{H} = \sqrt{\frac{\mu_0}{\varepsilon_0}} \overline{H}.
\end{equation*}

\subsection{Choice of the time and spatial step}
The choice of time step is driven by the spatial step and CFL condition. Assuming in general  $\Delta x = \Delta y = \Delta z$ 
\begin{equation*}
	\Delta t \leq \frac{\Delta x}{|v_p|} \cdot \frac{1}{\sqrt{D}}
\end{equation*}
the choice of the spatial step is provided by 
\begin{equation*}
	\Delta x \leq \frac{1}{m_\textrm{ovs}} \min (\lambda_\textrm{min},d_\textrm{min})
\end{equation*}
where $m_\textrm{ovs}$ represents the oversampling ratio in spatial domain. Typical values for $m_\textrm{ovs}$ are between 10 and 100. $\lambda_\textrm{min}$ is the smallest wavelength in our system. Generally it is defined as 
\begin{equation*}
	\lambda_\textrm{min} = \frac{v_p}{f_\textrm{max}} = \frac{c_0}{\sqrt{\varepsilon_{r,\textrm{min}}\mu_{r,\textrm{min}}}} \cdot f_\textrm{max}
\end{equation*}
where $v_p$ is the smallest velocity of the wave in any medium of our system, $d_\textrm{min}$ represents the smallest feature dimension we need to simulate and $f_\textrm{max}$ is the maximum frequency of the stimulus.